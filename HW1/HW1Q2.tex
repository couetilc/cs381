\documentclass[letterpaper,11pt]{article}
\usepackage[utf8]{inputenc}

\usepackage[margin=1in]{geometry} %One inch margins

%Makes things a little prettier
\usepackage{lmodern}%
\usepackage[T1]{fontenc}%
\usepackage{microtype}%
\usepackage[varqu,varl]{inconsolata} % sans serif typewriter

% Bring in the math!
\usepackage{amsmath}%
\usepackage{amsthm}%
\usepackage{amssymb}%
\usepackage{amsfonts}%

% mleftright is helpful for making the parenthesis (and other delimiters) fit their contents height-wise.
\usepackage{mleftright}%
% The following macro puts parenthesis around the contents using mleftright and adapt the height.
\newcommand{\parof}[1]{\mleft( #1 \mright)}
% Of course you can define similar commands for other delimiters.



% To typeset algorithms I simply use numbered lists (e.g., enumerate). I use nested lists for inner clauses. The following is a prettier version of this approach that I use in my notes; it changes the font of the numbering to typewriter font (as a visual cue), normalizes some margins, and so forth. I often put the whole algorithm inside a \begin{quote}...\end{quote} to help set it apart from the text. You can use labels and references (as usual) to refer to steps.

\usepackage{enumitem}%

\newcommand{\stepsfont}{\normalfont}%

\usepackage{enumitem}
\newlist{steps}{enumerate}{9}
\setlist[steps]{before=\upshape\stepsfont,font=\ttfamily} %
\setlist[steps,1]{leftmargin=*,labelsep=1ex,label={\arabic*.},ref={\arabic*}} %
\setlist[steps,2]{leftmargin=*,labelsep=1ex,label={\Alph*.},ref={\thestepsi.\Alph*}} %
\setlist[steps,3]{leftmargin=*,labelsep=1ex,label={\arabic*.},
  ref={\thestepsii.\arabic*}} %
\setlist[steps,4]{leftmargin=*,labelsep=1ex,label={\alph*.},
  ref={\thestepsiii.\alph*}} %
\setlist[steps,5]{leftmargin=*,labelsep=1ex,label={\arabic*.},
  ref={\thestepsiv.\arabic*}} %
\setlist[steps,6]{leftmargin=*,labelsep=1ex,label={\Alph*.},
  ref={\thestepsv.\Alph*}} %
\setlist[steps,7]{leftmargin=*,labelsep=1ex,label=\arabic*.,
  ref={\thestepsvi.\roman*}} %
\setlist[steps,8]{leftmargin=*,labelsep=1ex,label={\alph*.},
  ref={\thestepsvii.\alph*}} %
\setlist[steps,9]{leftmargin=*,labelsep=1ex,label={\arabic*.},
  ref={\thestepsviii.\arabic*}} %
\newlist{stepitems}{itemize}{1} % For when I want to put an itemized list inside the code.
\setlist[stepitems]{before=\upshape\ttfamily,label=*,leftmargin=*,labelsep=1ex}%



\title{CS381 Homework 1 Problem 2}
\author{Connor Couetil}
\date{\today}

\begin{document}

\maketitle

\section{Exercise 3.7}

The intuition is to compute pairs of sums, and then to realize that recursively
computing pairs of sums gives you the even prefix sums, after which the odd
prefix sums can be computed by simple subtraction of the previous value

\begin{quote}
    \noindent\texttt{\underline{prefix-sum($A$)}}: Given a array $A$ of length $n$, it returns an array of length $n$ containing the prefix sums of $A$.
    
    \begin{steps}
        \item If $|A|$ is 1, return $A$
        \item Let $B$ be an array of length $ \lfloor n / 2 \rfloor$
        \item $B[i] = a_{2i} + a_{2i-1}$ for $i = 1, ..., \lfloor n / 2 \rfloor$
        \item $C = $ \texttt{prefix-sum($B$)}
        \item Let $D$ be a new array of length $n$
        \item for $i = 1, ..., n$
        \begin{steps}
            \item if $i$ is even, $D[i] = C[i/2]$
            \item if $i$ is odd, $D[i] = C[\lceil i / 2 \rceil] - A[i]$
        \end{steps}
        \item return $D$
    \end{steps}
\end{quote}

The recurrence relation for work will be notated with $W$, the recurrence
relation for time will be notated with $T$.

The total work performed is $O(n \, log(n))$

\begin{quote}
$W(1) = 1$
\newline
$W(n) = O(n) + W(n / 2) + O(n)$
\end{quote}

The total time taken is $O(log(n))$

\begin{quote}
$T(1) = 1$
\newline
$T(n) = O(1) + T(n / 2) + O(1)$
\end{quote}

The summations can be done in parallel, thus $O(n)$ work is done in $O(1)$ time.
The size of the array is cut down by half at each iteration, thus the height of
the tree is $log(n)$ with constant time taken at each level.

\end{document}