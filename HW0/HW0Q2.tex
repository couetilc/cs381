\documentclass[letterpaper,11pt]{article}
\usepackage[utf8]{inputenc}

\usepackage[margin=1in]{geometry} %One inch margins

%Makes things a little prettier
\usepackage{lmodern}%
\usepackage[T1]{fontenc}%
\usepackage{microtype}%
\usepackage[varqu,varl]{inconsolata} % sans serif typewriter

% Bring in the math!
\usepackage{amsmath}%
\usepackage{amsthm}%
\usepackage{amssymb}%
\usepackage{amsfonts}%

% mleftright is helpful for making the parenthesis (and other delimiters) fit their contents height-wise.
\usepackage{mleftright}%
% The following macro puts parenthesis around the contents using mleftright and adapt the height.
\newcommand{\parof}[1]{\mleft( #1 \mright)}
% Of course you can define similar commands for other delimiters.



% To typeset algorithms I simply use numbered lists (e.g., enumerate). I use nested lists for inner clauses. The following is a prettier version of this approach that I use in my notes; it changes the font of the numbering to typewriter font (as a visual cue), normalizes some margins, and so forth. I often put the whole algorithm inside a \begin{quote}...\end{quote} to help set it apart from the text. You can use labels and references (as usual) to refer to steps.

\usepackage{enumitem}%

\newcommand{\stepsfont}{\normalfont}%

\usepackage{enumitem}
\newlist{steps}{enumerate}{9}
\setlist[steps]{before=\upshape\stepsfont,font=\ttfamily} %
\setlist[steps,1]{leftmargin=*,labelsep=1ex,label={\arabic*.},ref={\arabic*}} %
\setlist[steps,2]{leftmargin=*,labelsep=1ex,label={\Alph*.},ref={\thestepsi.\Alph*}} %
\setlist[steps,3]{leftmargin=*,labelsep=1ex,label={\arabic*.},
  ref={\thestepsii.\arabic*}} %
\setlist[steps,4]{leftmargin=*,labelsep=1ex,label={\alph*.},
  ref={\thestepsiii.\alph*}} %
\setlist[steps,5]{leftmargin=*,labelsep=1ex,label={\arabic*.},
  ref={\thestepsiv.\arabic*}} %
\setlist[steps,6]{leftmargin=*,labelsep=1ex,label={\Alph*.},
  ref={\thestepsv.\Alph*}} %
\setlist[steps,7]{leftmargin=*,labelsep=1ex,label=\arabic*.,
  ref={\thestepsvi.\roman*}} %
\setlist[steps,8]{leftmargin=*,labelsep=1ex,label={\alph*.},
  ref={\thestepsvii.\alph*}} %
\setlist[steps,9]{leftmargin=*,labelsep=1ex,label={\arabic*.},
  ref={\thestepsviii.\arabic*}} %
\newlist{stepitems}{itemize}{1} % For when I want to put an itemized list inside the code.
\setlist[stepitems]{before=\upshape\ttfamily,label=*,leftmargin=*,labelsep=1ex}%



\title{CS381 Homework 0 Problem 2}
\author{Connor Couetil}
\date{\today}

\begin{document}

\maketitle

\section{Exercise 2.1.2}

\paragraph{Recursive Spec}

The spec is CyclicHanoi($n$, $A$, $B$, $C$) where rings are moved from $A$ to $B$ stacked in sorted order.

\paragraph{Recursive Algorithm}

\begin{quote}
    \noindent\texttt{\underline{CyclicHanoi($n$, $A$, $B$, $C$)}}
    \begin{steps}
        \item If $n$ > 0
        \begin{steps}
            \item CyclicHanoi($n - 1$, $A$, $C$, $B$)
            \item Move top ring from $A$ to $B$
            \item CyclicHanoi($n-1$, $C$, $B$, $A$)
        \end{steps}
    \end{steps}
\end{quote}

\section{Exercise 2.1.3}

\paragraph{Recursive Spec}

The spec is DoubleCyclicHanoi($n$, $A$, $B$, $C$) where rings are moved from $A$ to $C$ stacked in sorted order.

\paragraph{Recursive Algorithm}

\begin{quote}
    \noindent\texttt{\underline{DoubleCyclicHanoi($n$, $A$, $B$, $C$)}}
    \begin{steps}
        \item CyclicHanoi($n$, $A$, $B$, $C$)
        \item CyclicHanoi($n$, $B$, $C$, $A$)
    \end{steps}
\end{quote}

\section{Exercise 2.1.4}

\paragraph{Recursive Spec}

The spec is ThickHanoi($3n$, $A$, $B$, $C$) where all $3n$ rings are moved from $A$ to $B$ stacked in sorted order.

\paragraph{Recursive Algorithm}

\begin{quote}
    \noindent\texttt{\underline{ThickHanoi($3n$, $A$, $B$, $C$)}}
    \begin{steps}
        \item If $n$ > 0
        \begin{steps}
            \item ThickHanoi($3(n-1)$, $A$, $C$, $B$)
            \item Move top ring from $A$ to $B$
            \item Move top ring from $A$ to $B$
            \item Move top ring from $A$ to $B$
            \item ThickHanoi($3(n-1)$, $C$, $B$, $A$)
        \end{steps}
    \end{steps}
\end{quote}

\section{Exercise 2.1.5}

\paragraph{Recursive Spec}

The spec is TripleHanoi($3n$, $A$, $B$, $C$) where $3n$ rings are moved such that $n$ distinct rings are on each post, one of each size, stacked in sorted order.

\paragraph{Recursive Algorithm}

\begin{quote}
    \noindent\texttt{\underline{TripleHanoi($3n$, $A$, $B$, $C$)}}
    \begin{steps}
        \item If $n$ > 0
        \begin{steps}
            \item Towers-of-Hanoi($3n-1$, $A$, $B$, $C$)
            \item Towers-of-Hanoi($3n-2$, $B$, $C$, $A$)
            \item Towers-of-Hanoi($3n-3$, $C$, $A$, $B$)
            \item TripleHanoi($3(n-1)$, $A$, $B$, $C$)
        \end{steps}
    \end{steps}
\end{quote}


\section{Exercise 2.1.6}

\paragraph{Recursive Spec}

The spec is AmericanHanoi($n$, $Red$, $White$, $Blue$) where $n$ colored rings are moved to the post with the matching color stacked in sorted order.

\paragraph{Recursive Algorithm}

\begin{quote}
    \noindent\texttt{\underline{AmericanHanoi($n$, $Red$, $White$, $Blue$)}}
    \begin{steps}
        \item If $n > 0$
        \begin{steps}
            \item If bottom ring of $Red$ is white
            \begin{steps}
                \item Towers-of-Hanoi($n-1$, $Red$, $Blue$, $White$)
                \item Move top ring of $Red$ to $White$
                \item Towers-of-Hanoi($n-1$, $Blue$, $Red$, $White$)
            \end{steps}
            \item If bottom ring of $Red$ is blue
            \begin{steps}
                \item Towers-of-Hanoi($n-1$, $Red$, $White$, $Blue$)
                \item Move top ring of $Red$ to $Blue$
                \item Towers-of-Hanoi($n-1$, $White$, $Red$, $Blue$)
            \end{steps}
            \item AmericanHanoi($n-1$, $Red$, $White$, $Blue$)
        \end{steps}
    \end{steps}
\end{quote}

\section{Exercise 2.1.8}

My variant is called "AustralianHanoi".

\paragraph{Recursive Spec}

The spec is AustralianHanoi($n$, $A$, $B$, $C$) where $n$ rings are moved from $A$ to $B$ in the reverse order they started in. Crikey!

\paragraph{Recursive Algorithm}

\begin{quote}
    \noindent\texttt{\underline{AustralianHanoi($n$, $A$, $B$, $C$)}}
    \begin{steps}
        \item If $n > 0$
        \begin{steps}
            \item Move the top ring from $A$ to $B$
            \item AustralianHanoi($n - 1$, $A$, $B$, $C$)
        \end{steps}
    \end{steps}
\end{quote}

\end{document}