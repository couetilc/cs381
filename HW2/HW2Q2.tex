\documentclass[letterpaper,11pt]{article}
\usepackage[utf8]{inputenc}

\usepackage[margin=1in]{geometry} %One inch margins

%Makes things a little prettier
\usepackage{lmodern}%
\usepackage[T1]{fontenc}%
\usepackage{microtype}%
\usepackage[varqu,varl]{inconsolata} % sans serif typewriter
\usepackage[dvipsnames]{xcolor}

% Bring in the math!
\usepackage{amsmath}%
\usepackage{amsthm}%
\usepackage{amssymb}%
\usepackage{amsfonts}%

% mleftright is helpful for making the parenthesis (and other delimiters) fit their contents height-wise.
\usepackage{mleftright}%
% The following macro puts parenthesis around the contents using mleftright and adapt the height.
\newcommand{\parof}[1]{\mleft( #1 \mright)}
% Of course you can define similar commands for other delimiters.



% To typeset algorithms I simply use numbered lists (e.g., enumerate). I use nested lists for inner clauses. The following is a prettier version of this approach that I use in my notes; it changes the font of the numbering to typewriter font (as a visual cue), normalizes some margins, and so forth. I often put the whole algorithm inside a \begin{quote}...\end{quote} to help set it apart from the text. You can use labels and references (as usual) to refer to steps.

\usepackage{enumitem}%

\newcommand{\stepsfont}{\normalfont}%

\usepackage{enumitem}
\newlist{steps}{enumerate}{9}
\setlist[steps]{before=\upshape\stepsfont,font=\ttfamily} %
\setlist[steps,1]{leftmargin=*,labelsep=1ex,label={\arabic*.},ref={\arabic*}} %
\setlist[steps,2]{leftmargin=*,labelsep=1ex,label={\Alph*.},ref={\thestepsi.\Alph*}} %
\setlist[steps,3]{leftmargin=*,labelsep=1ex,label={\arabic*.},
  ref={\thestepsii.\arabic*}} %
\setlist[steps,4]{leftmargin=*,labelsep=1ex,label={\alph*.},
  ref={\thestepsiii.\alph*}} %
\setlist[steps,5]{leftmargin=*,labelsep=1ex,label={\arabic*.},
  ref={\thestepsiv.\arabic*}} %
\setlist[steps,6]{leftmargin=*,labelsep=1ex,label={\Alph*.},
  ref={\thestepsv.\Alph*}} %
\setlist[steps,7]{leftmargin=*,labelsep=1ex,label=\arabic*.,
  ref={\thestepsvi.\roman*}} %
\setlist[steps,8]{leftmargin=*,labelsep=1ex,label={\alph*.},
  ref={\thestepsvii.\alph*}} %
\setlist[steps,9]{leftmargin=*,labelsep=1ex,label={\arabic*.},
  ref={\thestepsviii.\arabic*}} %
\newlist{stepitems}{itemize}{1} % For when I want to put an itemized list inside the code.
\setlist[stepitems]{before=\upshape\ttfamily,label=*,leftmargin=*,labelsep=1ex}%



\title{CS381 Homework 2 Problem 1}
\author{Connor Couetil}
\date{\today}

\begin{document}

\maketitle

\section{Exercise 5.2.1}

\begin{quote}
    \noindent\texttt{\underline{longest-common-sub($i$, $k$)}}: Given $A[1, ...,
    m]$ and $B[1, ..., n]$ as globals where parameter $i \in [m]$ and $k \in
    [n]$ , returns the length of the longest common subsequence of $A[1, ...,
    i]$ and $B[1, ..., k]$.

    \begin{steps}
        \item If $i = 0$ or $k = 0$ Return 0
        \item If $A[i] = A[k]$
        \begin{steps}
            \item Return $Max(
                \texttt{longest-common-sub($i-1$, $k$)},
                \texttt{longest-common-sub($i$, $k-1$)},
                2 + \texttt{longest-common-sub($i-1$, $k-1$)},
                )$
        \end{steps}
        \item Return $Max(
            \texttt{longest-common-sub($i-1$, $k$)},
            \texttt{longest-common-sub($i$, $k-1$)},
            )$
    \end{steps}
\end{quote}

Given arrays $A$ with length $m$ and $B$ with length $n$, find the longest
common subsequence with the call $\texttt{longest-common-sub($m$, $n$)}$.  The
number of possible parameter combinations is $m \cdot n$, thus the runtime is
$O(mn)$ when the function is memoized based on parameters $i$ and $k$

\section{Exercise 5.2.2}

\begin{quote}
    \noindent\texttt{\underline{shortest-common-super($i$, $k$)}}: Given $A[1, ...,
    m]$ and $B[1, ..., n]$ as globals where parameter $i \in [m]$ and $k \in
    [n]$ , returns the length of the shortest common supersequence of $A[1, ...,
    i]$ and $B[1, ..., k]$.

    \begin{steps}
        \item If $i = 0$ Return $n$ \textcolor{ForestGreen}{// |B|}
        \item If $k = 0$ Return $m$ \textcolor{ForestGreen}{// |A|}
        \item If $A[i] = B[k]$
        \begin{steps}
            \item Return $1 + \texttt{shortest-common-super($i-1$, $k-1$)}$
        \end{steps}
        \item Return $Min(1 + \texttt{shortest-common-super($i-1$, $k$)}, 1 +
        \texttt{shortest-common-super($i$, $k-1$)})$
    \end{steps}
\end{quote}

Given arrays $A$ with length $m$ and $B$ with length $n$, find the shortest
common supersequence with the call $\texttt{shortest-common-super($m$, $n$)}$.  The
number of possible parameter combinations is $m \cdot n$, thus the runtime is
$O(mn)$ when the function is memoized based on parameters $i$ and $k$.

\section{Exercise 5.2.3}

\begin{quote}
    \noindent\texttt{\underline{palindrome($i$, $j$, $k$, $l$)}}: Given $A[1,
    ..., m]$ and $B[1, ..., n]$ as globals where parameter $i,j \in [m]$ and
    $k,l \in [n]$ , returns the length of the longest common subsequence of
    $A[1, ..., i]$ and $B[1, ..., k]$ that is a palindrome.

    \begin{steps}
        \item If $i > j$ or $k > l$ Return 0
        \item If $i = j$
        \begin{steps}
            \item If $A[i] = B[k]$ or $A[i] = B[l]$
            \begin{steps}
                \item Return 1
            \end{steps}
            \item Else
            \begin{steps}
                \item Return 0
            \end{steps}
        \end{steps}
        \item If $k = l$
        \begin{steps}
            \item If $B[k] = A[i]$ or $B[k] = A[j]$
            \begin{steps}
                \item Return 1
            \end{steps}
            \item Else 
            \begin{steps}
                \item Return 0
            \end{steps}
        \end{steps}
        \item If $A[i] = A[j] = B[k] = B[l]$
        \begin{steps}
            \item Return $Max($
                2 + \texttt{palindrome($i+1$, $j-1$, $k+1$, $l-1$)},
                \texttt{palindrome($i+1$, $j$, $k$, $l$)},
                \texttt{palindrome($i$, $j-1$, $k$, $l$)},
                \texttt{palindrome($i$, $j$, $k+1$, $l$)},
                \texttt{palindrome($i$, $j$, $k$, $l-1$)}
                $)$
        \end{steps}
        \item Else
        \begin{steps}
            \item $Max($
                \texttt{palindrome($i+1$, $j$, $k$, $l$)},
                \texttt{palindrome($i$, $j-1$, $k$, $l$)},
                \texttt{palindrome($i$, $j$, $k+1$, $l$)},
                \texttt{palindrome($i$, $j$, $k$, $l-1$)}
                $)$
        \end{steps}
    \end{steps}
\end{quote}

Given arrays $A$ with length $m$ and $B$ with length $n$, find the longest
common subsequence that is a palindrome with the call $\texttt{palindrome($0$,
$m$, $0$, $n$)}$.  The number of possible parameter combinations is $m \cdot m
\cdot n \cdot n$, thus the runtime is $O(m^2n^2)$ when the function is memoized
based on parameters $i$, $j$, $k$ and $l$.

\end{document}