\documentclass[letterpaper,11pt]{article}
\usepackage[utf8]{inputenc}

\usepackage[margin=1in]{geometry} %One inch margins

%Makes things a little prettier
\usepackage{lmodern}%
\usepackage[T1]{fontenc}%
\usepackage{microtype}%
\usepackage[varqu,varl]{inconsolata} % sans serif typewriter
\usepackage[dvipsnames]{xcolor}

% Bring in the math!
\usepackage{amsmath}%
\usepackage{amsthm}%
\usepackage{amssymb}%
\usepackage{amsfonts}%

% mleftright is helpful for making the parenthesis (and other delimiters) fit their contents height-wise.
\usepackage{mleftright}%
% The following macro puts parenthesis around the contents using mleftright and adapt the height.
\newcommand{\parof}[1]{\mleft( #1 \mright)}
% Of course you can define similar commands for other delimiters.



% To typeset algorithms I simply use numbered lists (e.g., enumerate). I use nested lists for inner clauses. The following is a prettier version of this approach that I use in my notes; it changes the font of the numbering to typewriter font (as a visual cue), normalizes some margins, and so forth. I often put the whole algorithm inside a \begin{quote}...\end{quote} to help set it apart from the text. You can use labels and references (as usual) to refer to steps.

\usepackage{enumitem}%

\newcommand{\stepsfont}{\normalfont}%

\usepackage{enumitem}
\newlist{steps}{enumerate}{9}
\setlist[steps]{before=\upshape\stepsfont,font=\ttfamily} %
\setlist[steps,1]{leftmargin=*,labelsep=1ex,label={\arabic*.},ref={\arabic*}} %
\setlist[steps,2]{leftmargin=*,labelsep=1ex,label={\Alph*.},ref={\thestepsi.\Alph*}} %
\setlist[steps,3]{leftmargin=*,labelsep=1ex,label={\arabic*.},
  ref={\thestepsii.\arabic*}} %
\setlist[steps,4]{leftmargin=*,labelsep=1ex,label={\alph*.},
  ref={\thestepsiii.\alph*}} %
\setlist[steps,5]{leftmargin=*,labelsep=1ex,label={\arabic*.},
  ref={\thestepsiv.\arabic*}} %
\setlist[steps,6]{leftmargin=*,labelsep=1ex,label={\Alph*.},
  ref={\thestepsv.\Alph*}} %
\setlist[steps,7]{leftmargin=*,labelsep=1ex,label=\arabic*.,
  ref={\thestepsvi.\roman*}} %
\setlist[steps,8]{leftmargin=*,labelsep=1ex,label={\alph*.},
  ref={\thestepsvii.\alph*}} %
\setlist[steps,9]{leftmargin=*,labelsep=1ex,label={\arabic*.},
  ref={\thestepsviii.\arabic*}} %
\newlist{stepitems}{itemize}{1} % For when I want to put an itemized list inside the code.
\setlist[stepitems]{before=\upshape\ttfamily,label=*,leftmargin=*,labelsep=1ex}%



\title{CS381 Homework 2 Problem 1}
\author{Connor Couetil}
\date{\today}

\begin{document}

\maketitle

\section{Exercise 5.13}

\begin{quote}
    \noindent\texttt{\underline{subs($k$, $h$, $v$)}}: Given parameters $k$, a
    set of bids (a bid is a tuple of $x$, the order price, and $y$, the inches of
    sub that fulfills the order), $h$, the inches of ham sub left, and $v$, the
    inches of veggie sub left, returns the maximum profit from selling the two
    sandwiches to the
    set of bidders.
    \begin{steps}
        \item If $k$ is empty, Return $0$
        \item $b = k[0]$
        \item if $b$ is a bid for ham and $h \geq b_y$
        \begin{steps}
            \item Return $Max($
            \\ \hphantom{    } $b_x - \texttt{cost($b_y$)} + \texttt{subs($k \setminus b$, $h-b_y$, $v$)}$, \textcolor{ForestGreen}{// Take the bid}
            \\ \hphantom{    } $\texttt{subs($k \setminus b$, $h$, $v$)}$ \textcolor{ForestGreen}{// Don't take the bid}
            \\ $)$
        \end{steps}
        \item if $b$ is a bid for veggie and $v \geq b_y$
        \begin{steps}
            \item Return $Max($
            \\ \hphantom{    } $b_x - \texttt{cost($b_y$)} + \texttt{subs($k \setminus b$, $h$, $v-b_y$)}$, \textcolor{ForestGreen}{// Take the bid}
            \\ \hphantom{    } $\texttt{subs($k \setminus b$, $h$, $v$)}$ \textcolor{ForestGreen}{// Don't take the bid}
            \\ $)$
        \end{steps}
        \item If $b$ is disjunctive and $h \geq b_y$ and $v \geq b_y$
        \begin{steps}
            \item Return $Max($
            \\ \hphantom{    } $b_x - \texttt{cost($b_y$)} + \texttt{subs($k \setminus b$, $h-b_y$, $v$)}$, \textcolor{ForestGreen}{// Take the ham bid}
            \\ \hphantom{    } $b_x - \texttt{cost($b_y$)} + \texttt{subs($k \setminus b$, $h$, $v-b_y$)}$, \textcolor{ForestGreen}{// Take the veggie bid}
            \\ \hphantom{    } $\texttt{subs($k \setminus b$, $h$, $v$)}$ \textcolor{ForestGreen}{// Don't take the bid}
            \\ $)$
        \end{steps}
        \item If $b$ is disjunctive and $h \geq b_y$
        \begin{steps}
            \item Return $Max($
            \\ \hphantom{    } $b_x - \texttt{cost($b_y$)} + \texttt{subs($k \setminus b$, $h-b_y$, $v$)}$, \textcolor{ForestGreen}{// Take the ham bid}
            \\ \hphantom{    } $\texttt{subs($k \setminus b$, $h$, $v$)}$ \textcolor{ForestGreen}{// Don't take the bid}
            \\ $)$
        \end{steps}
        \item If $b$ is disjunctive and $v \geq b_y$
        \begin{steps}
            \item Return $Max($
            \\ \hphantom{    } $b_x - \texttt{cost($b_y$)} + \texttt{subs($k \setminus b$, $h$, $v-b_y$)}$, \textcolor{ForestGreen}{// Take the veggie bid}
            \\ \hphantom{    } $\texttt{subs($k \setminus b$, $h$, $v$)}$ \textcolor{ForestGreen}{// Don't take the bid}
            \\ $)$
        \end{steps}
        \item Return $\texttt{subs($k \setminus b$, $h$, $v$)}$ \textcolor{ForestGreen}{// Unable to take bid}
    \end{steps}
\end{quote}

Given a ham sub $m$ inches long and a veggie sub $n$ inches long, find the
maximum profit from selling the subs to a set of $k$ bidders by calling
\texttt{subs($k$, $m$, $n$)}. The total number of calls that will be made is the
size of the powerset of the set of bids, so the runtime is $O(2^k)$.

\end{document}